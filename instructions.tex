\chapter{Instructions}
\section{Introduction} % sections break up the document into pieces
These instructions detail how to prepare a final project report, master's thesis, or PhD dissertation for Reykjavík University.
These instructions (unless otherwise stated) assume you are in the School of Science and Engineering or the School of Computer Science.
If you are in another department, you should make sure that the template meets your specific requirements.



\begin{itemize}
\item Current maintainer: Joseph Timothy Foley.
  Questions, comments, complaints:  \formatemail{foley AT ru.is}
% or cs-grad AT ru.is  %% not sure why we had this address

%\item To receive updates regarding the templates, subscribe at
%  \url{https://list.ru.is/mailman/listinfo/latex-announcements}
\end{itemize}


\section{Where to get the files}
\begin{description}
\item [Actively developed code:] \url{https://github.com/ru-engineering/report-template}
\item [Overleaf Template:]  \textsc{Coming Soon}.
\end{description}


\section{Files and Directories/Folders}
\begin{itemize}
\item \path{graphics-crop/:} is a place to put smaller versions of images to reduce the produced file size (and speed up printing)
\item \path{graphics/:} contains the main graphics to generate this document.
\item \path{covers/:} contains the official covers (from RU Communications) to be put on the outside of the finished book.
\end{itemize}

\section{LaTeX Template Instructions}
Some information is at the top of \path{main.tex} file, this file is for a general overview and common problems.
This content is in the \path{instructions.tex} file and should be commented out of the \path{main.tex} file once you have begun putting your content into the template.
We recommend keeping the file around in case you need to look things up.

\subsection{Getting started:}
\begin{enumerate}
\item Find a safe place to work on your thesis document.
  The author recommends Git on Overleaf, but anywhere data is backed up is a appropriate.
  If you wish to have a repository to be setup for your thesis on \url{openproject.cs.ru.is}, email csit AT ru.is.  
  If you are working with sensitive information, you should avoid bitbucket, google drive, dropbox, and any other free cloud service.
  If you think this is unnecessary, just consider how much time you will lose if your computer crashes.
   Due to Murphy's law, this is likely to happen just before your thesis is due\footnote{This has happened many times.}.

 \item Get a LaTeX installation.  We recommend TeXlive \url{https://www.tug.org/texlive/}
   For this template on windows, MiKTeX will also work, but will run very slowly the first time you render the template.
   You will need to enable the ``miktex'' option in the template to substitute packages.
   It is very very important that you run the ``MikTeX Update Wizard'' before you start.
   Otherwise you may get errors when you try to build the document.

   Under linux this is the ``texlive'' package.
   Under Mac/OSX this is the ``MacTeX'' distribution.

   Alternatively, if nothing you are doing is particularly private or proprietary, you can do development online using Overleaf.
   In this case, you won't need to setup the rest of the tools mentioned below except perhaps the Reference Manager mentioned in step~\ref{list:refmanager}.
   

   \begin{enumerate}
   \item RedHat: sudo yum -y install
     texlive-collection-fontsrecommended
     texlive-biblatex-{apa,apa-doc,ieee,ieee-doc}
     texlive-{xargs,lipsum,lastpage,luatex,pseudocode,url,examplep,listings,xspace,pgf,tikz,amsfonts,amsmath,amssymb,siunitx,svn-multi,subfig,fixme,textpos,biblatex,makeglos,nomencl,xwatermark,ltxkeys,framed,boondox,printlen}
     Getting biber installed on older RedHat systems is a bit tricky
     for unclear reasons.  The metapackage you need is at
     https://copr.fedoraproject.org/coprs/cbm/Biber/ 
   \item Debian/Ubuntu:
     sudo apt-get -y install texlive-full pgf latex-xcolor
     If you don't want to install everything, this list of packages is known
     to work: sudo apt-get -y install texlive texlive-luatex texlive-latex-extra
     texlive-science texlive-generic-extra texlive-lang-european
     texlive-lang-german latex-xcolor texlive-pictures pgf
     texlive-bibtex-extra texlive-publishers chktex evince
     fonts-lmodern lmodern biber
   \end{enumerate}
 \item Get a LaTeX Integrated Development Environment (recommended, but not required)
   \url{http://texstudio.sourceforge.net/} or
   \url{http://www.xm1math.net/texmaker/}
   Some editors may include LaTeX support.
   If you want to learn a very powerful (but old-fashioned) editor \url{http://www.gnu.org/software/emacs/}
      Install the auctex package by: M-x list-packages, click on AUCTeX

    \item Get a references manager (recommended, but not required)\label{list:refmanager}
   \url{http://jabref.sourceforge.net/}  (You may have to install a Java JRE first.)
   The reference library is in \path{references.bib} by default.
   It is just a text file that can be edited, but be careful with the formatting.
   A common mistake is to forget ``,'' at the end of each piece of an entry/line.

   If you are going to make glossaries or acronym lists, you will need
   a perl interpreter.  Only windows usually needs this installed:
   \url{http://www.activestate.com/activeperl}

 \item Get supporting programs for some tools.
   For glossaries under windows, you will need to install Perl
   \url{http://strawberryperl.com/}
   (it is already installed on the other platforms.)
   
 \item Try building the \path{main.tex} file.  If you get errors, there
   is something wrong with your LaTeX installation.  Fix those first.

   
 \item Rename the \path{main.tex} file with your information (optional).
   DEGREE-NAME-YEAR is the recommended scheme 
   e.g. \path{msc-foley-2015.tex}.
   This is referred to as the "Main" file.
 \item Set your UI to use \path{lualatex} as the processor.
   If you are typing commands in manually, this is by typing in \path{lualatex main.tex}
   
 \item Open and read the options at the top of the previous file and set
   it up for your document.
   You will need to fill in the title and author at least.

 \item Start editing all of the \path{.tex} files with your content.

 \item Compile the document by running lualatex on the Main file, run the bibliography tool, then view the result.

 \item When you print, make sure that the scale is 100\%.
   If you allow it to resize when printing, the margins won't be right.
   If the  margins aren't right, then the RU logo will not look right on the
   cover.

\end{enumerate}

\subsection{Important Details}

\begin{itemize}
\item Make absolutely sure that your \path{references.bib} is in UTF-8.  If it is another format (CP1251,etc) you may get weird problems with any accented characters.
  {\em Students have run into encoding issues in the past and it has taken a surprisingly long time to debug.}
\item Make sure the rest of the files, particularly the \path{.tex} file are in UTF8 or are at least in the same encoding.
  If the files are in different encoding, you will discover errors with accented characters when you try to include them together.
  Watch out for line endings. Linux, Windows, and OSX all use different line endings in text files.

\item You may wish biber/biblatex instead of bibtex.
(The template may already do this.)
Otherwise Icelandic characters may not work properly in your \path{references.bib} file.
TexMaker and TeXStudio require a configuration change to do this.
%Refer to the ``...Projects'' guide above.

\item Be consistent about UPPER and lower case in naming files.
  Windows doesn't care (but some programs in Windows do).
  OSX sometimes cares.
  Linux always cares.
\item When using this template with SVN, you will want to tell it to ignore the extensions listed in Appendix~\ref{appendix:latex-gen}
\end{itemize}


\subsection{LaTeX Generated file extensions}\label{appendix:latex-gen}
These are the files that \LaTeX{} generates when you run it.
If you are using SVN or another version control system, you want to tell that system to ignore these files:
  \begin{verbatim}
*-blx.bib
*.acr
*.acn
*.alg
*.aux
*.bak
*.bbl
*.bcf
*.blg
*.bst
*.dvi
*.glo
*.gl*
*.idx
*.ind
*.ilg
*.ist
*.lo?
*.mw
*.nlo
*.ntn
*.out
*.pdf
*.ps
*.rel
*.run.xml
*.sbl
*.slg
*.snm
*.sym
*.synctex.gz
*.tcp
*.thm
*.tdo
*.to?
*.tmp
*.tmproj
*.xwm
._*
._.DS_Store
.~lock*
auto
Thumbs.db
\end{verbatim}

%\end{document}

%%% Local Variables:
%%% mode: latex
%%% TeX-master: "main.tex"
%%% TeX-engine: luatex
%%% End:
