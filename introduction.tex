\chapter{Introduction\label{cha:introduction}}

The introduction starts with an expansion of the first half of the abstract where you introduct context, stakeholder, and the topic.

In this domain and topic, what opportunities could there be?  (EngX Week 2, Chapter 3~\cite{ulrich2020product-design-development})

Based upon the most promising opportunity, explain how it might be VRIN. (EngX Week 2, Chapter 3~\cite{ulrich2020product-design-development})

Because you want to explore the VRIN opporunity, justify that it is worth doing with Real-Win-Worth it. (EngX Week 2, Chapter 3~\cite{ulrich2020product-design-development})

\section{Stakeholder}
Write at least a page explaining the general context of what you are working.
What is the problem and who/what does it affect?
How much of an impact does this problem (or lack of a solution) have?
Tell us as much as you can about the stakeholder: (Week 2 in EngX, Chapter 5 in \cite{ulrich2020product-design-development}).
Are there other stakeholders who might also be affected?

\section{Background}
Provide background about the subject matter.
Think of this as a bit of a history lesson (e.g. How was morse code
developed?  How is it used today?.

This is a place where there are usually many citations.
It is suspicious when there is not.
If you have specifications or related standards, these must be
described and cited also.
As an example, you might cite the specific
RoboSub competition website (and documents) if working on the lighting system for an AUV\cite{guls2016auvlight}\index{AUV}

\section{Goals}
Write up the customer needs analysis here (Week 3 in EngX, Chapter 5 in~\cite{ulrich2020product-design-development}).
Develop a high-level strategy of how you will solve the stakeholder's problem by meeting the highest needs.
Which latent needs seem more interesting to explore?

\section{Prior Art}
What is the current state of the art in this topic?
What are the products, tools, and/or techniques?
How do they compare?

\subsection{Cool System 1}
This system from Cool Company is able to do cool things:  BLah blah blah.
You can learn more about this particular product at CITATION.
\subsection{Cool System 2}
\subsection{Cool System 3}

End this section with a competitive bench marking chart~\cite[p.104]{ulrich2020product-design-development})

\section{Example Section}
\begin{figure}
  \centering
  \includegraphics[width=0.3\textwidth]{ru-logo}
  \caption[RU Logo]{Reykjavík University Logo, used with permission.}\label{fig:ru-logo}
\end{figure}
\begin{table}
  \centering
  \begin{tabular}{ll}\toprule
    $x$& $x^{2}$\\\midrule
    1 &1\\
    2 &4\\
    3 &9\\\bottomrule
  \end{tabular}
  \caption{Table of squared numbers}\label{tab:numbers}
\end{table}
There is an RU logo in Figure~\ref{fig:ru-logo}.
This logo will scale according to the width of the text on the page.
There is a helpful list of squared numbers in Table~\ref{tab:numbers}.
This table is formatted in the style of a book, which is very differerent than the style one is used to in Excel.

The test text ``Lorem Ipsum''\index{Lorem Ipsum} is from an ancient text from 45 B.C. \cite{cicero46deFinibus, lipsomwebsite}\\
\lipsum[1-5]


\section{Data Management}
What data do you expect you need to collect electonically and how will you protect it following GDPR rules?  (EngX Week 3)
What regulations are relevant (cite them).
If the data is leaked, what will you do?

%%% Local Variables:
%%% mode: latex
%%% TeX-master: "main"
%%% End:
